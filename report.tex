\documentclass[a4j, titlepage]{jarticle}
\usepackage{amsmath,amssymb} % 数式
\usepackage{fancybox,ascmac} % 丸枠
\usepackage[dvipdfmx]{graphicx} % 図
\usepackage{verbatim} % ソースコードの埋め込み
% プログラムリストで使用
\usepackage{ascmac}
\usepackage{here}
\usepackage{txfonts}
\usepackage{listings, jlisting} % プログラムリスト
\renewcommand{\lstlistingname}{リスト}
\lstset{
  language=c,
  basicstyle=\ttfamily\small, % コードのフォントと文字サイズ
  commentstyle=\textit, % コメント部分のフォント
  classoffset=1,
  keywordstyle=\bfseries,
  frame=tRBl,
  framesep=5pt,
  showstringspaces=false,
  numbers=left,
  stepnumber=1,
  numberstyle=\footnotesize,
  tabsize=3 % インデントの深さ(スペースの数)
}
\title{ \Huge Markdown -\textgreater{} TeX [ -\textgreater{} PDF ] \\{\LARGE markdownの記述例}}
\author{ \Large @TeX2e }
\date{ \Large 2015/4/1 }

\begin{document}
\maketitle
\thispagestyle{empty}
\newpage
\setcounter{page}{1}

\if 0
:preamble
	{\tt \textbackslash{}def\textbackslash{}lstlistingname\{List\}}
	{\tt \textbackslash{}def\textbackslash{}tablename\{Table\}}

\fi
\section{Overview}

これはMarkdownファイルをTeXファイルに変換するためのrubyスクリプトです。
rubyが実行できる環境と、gemの''kramdown''が必要です。

\section{Usage}

\begin{screen}
\begin{verbatim}
ruby <this_script> <md_file> [-p]
\end{verbatim}
\end{screen}

\section{Structural Elements}

\begin{itemize}
\item Headers
\item Lists
\item Code Blocks
\item Tables
\item Math Blocks
\item Images
\end{itemize}

\section{Markdown Syntax}

\subsection{Headers}

見出しは {\tt \#} を使って表します。

\begin{itembox}[c]{見出しの例}
\begin{verbatim}
# First level header

## Second level header

### Third level header
\end{verbatim}
\end{itembox}

\subsection{(Un)Ordered Lists}

箇条書きには {\tt -,+,*} が使えます。
リストは {\tt 1.} のように数字とコロンと1つ以上の空白をリストの先頭に付けます。

\begin{itembox}[c]{list example}
\begin{verbatim}
- item1
- item2
- item3

1. item1
2. item2
3. item3
\end{verbatim}
\end{itembox}

\if 0
\begin{itemize}
\item item1
\item item2


\begin{itemize}
\item nest1
\item nest2


\begin{itemize}
\item deep nest1
\item deep nest2
\end{itemize}
\end{itemize}
\end{itemize}

\begin{enumerate}
\item item1
\item item2


\begin{enumerate}
\item nest1
\item nest2


\begin{enumerate}
\item deep nest1
\item deep nest2
\end{enumerate}
\end{enumerate}
\end{enumerate}

\fi
\subsection{Definition Lists}

定義の次の行に {\tt :} があれば、定義とその説明を書くことができます。

\begin{itembox}[c]{definition list example}
\begin{verbatim}
def1
: description

def2
: description
\end{verbatim}
\end{itembox}

\begin{description}
\item[Laziness]\mbox{}\\ The quality that makes you go to great effort to reduce overall energy expenditure. It makes you write labor-saving programs that other people will find useful, and document what you wrote so you don't have to answer so many questions about it. Hence, the first great virtue of a programmer.



\item[Impatience]\mbox{}\\ The anger you feel when the computer is being lazy. This makes you write programs that don't just react to your needs, but actually anticipate them. Or at least that pretend to. Hence, the second great virtue of a programmer.



\item[Hubris]\mbox{}\\ Excessive pride, the sort of thing Zeus zaps you for. Also the quality that makes you write (and maintain) programs that other people won't want to say bad things about. Hence, the third great virtue of a programmer.
\end{description}

\subsection{Code Blocks}

ソースコードを出力する方法

\begin{itemize}
\item ソースコードの前後に1つ以上の空行を置く
\item 4つ以上のインデントまたは1つ以上のタブを置く
\item {\tt :caption} でタイトルを付ける
\item {\tt :label} でラベルを付ける
\item {\tt :listing} で行番号と改ページを行う枠に変更する
\end{itemize}

ソースコードは、丸枠で囲むか、行番号付きの枠で囲むかの2通りの選択肢があります。

丸枠を使う方法は3通りあります。

\begin{itembox}[c]{タイトルなしの枠}
\begin{verbatim}
 
    printf("hello, world");
 
\end{verbatim}
\end{itembox}

\begin{itembox}[c]{丸枠の例}
\begin{verbatim}
:caption <caption>

    printf("hello, world");
 
\end{verbatim}
\end{itembox}

\begin{itembox}[c]{埋め込みの例}
\begin{verbatim}
:caption <caption>
    [embed](/path/to/source.c)
 
\end{verbatim}
\end{itembox}

行番号付きの枠を使う方法は2通りあります。

\begin{itembox}[c]{行番号付きの枠}
\begin{verbatim}
:caption <caption> :label <label>
:listing

    (1..10).each do |i|
        p i
    end
 
\end{verbatim}
\end{itembox}

\begin{itembox}[c]{埋め込みの例}
\begin{verbatim}
:caption <caption> :label <label>
:listing
    [embed](/path/to/source.c)
 
\end{verbatim}
\end{itembox}

行番号付きの枠の場合、{\tt :}{\tt ref\{\textless{}label\textgreater{}\}} で参照を行うことができます。

\subsubsection{Samples}

以下に出力例を示します。

\begin{screen}
\begin{verbatim}
printf("hello, world");
\end{verbatim}
\end{screen}

\begin{itembox}[c]{hello, world}
\begin{verbatim}
printf("hello, world");
\end{verbatim}
\end{itembox}

\begin{itembox}[c]{embed}
{\small
\verbatiminput{./sample.c}
}
\end{itembox}

\begin{lstlisting}[caption=iterate ,label=list:1]
(1..10).each do |i|
	p i
end
\end{lstlisting}

\lstinputlisting[caption=embed in list ,label=list:2]
{./sample.c}

list\ref{list:2} shows \ldots{}

\subsection{Tables}

表は、仕切りに {\tt -} と {\tt \textbar{}} を使って表します

\begin{itembox}[c]{table example}
\begin{verbatim}
:caption <caption> :label <label>

 colum1     | colum2      | colum3
:-----------|------------:|:------------:
 This       | This        | This         
 column     | column      | column       
 will       | will        | will         
 be         | be          | be           
 left       | right       | center       
 aligned    | aligned     | aligned   
\end{verbatim}
\end{itembox}

\begin{table}[h]
\centering
\caption{table sample }
\label{table:1}
\begin{tabular}{|l|r|c|}
\hline
Left align & Right align & Center align\\
\hline
This & This & This\\
column & column & column\\
will & will & will\\
be & be & be\\
left & right & center\\
aligned & aligned & aligned\\
\hline
\end{tabular}
\end{table}

table\ref{table:1} shows \ldots{}

\subsection{Math Blocks}

数式は\$\$で囲みます

\begin{itembox}[c]{数式の例}
\begin{verbatim}
$$ inline math block $$

$$
multiline math block
$$
\end{verbatim}
\end{itembox}

解の公式は $ x = \frac{-b\pm\sqrt{b^2-4ac}}{2a} $ で表せます。
式 $ \sum_{n = 1}^{\infty} \frac{1}{n} $ の収束値を求めます。

\begin{eqnarray*}
\frac{\pi}{2}
= \left( \int_{0}^{\infty} \frac{\sin x}{\sqrt{x}} dx \right)^2 
= \sum_{k=0}^{\infty} \frac{(2k)!}{2^{2k}(k!)^2} \frac{1}{2k+1} 
= \prod_{k=1}^{\infty} \frac{4k^2}{4k^2 - 1}
\end{eqnarray*}

\subsection{Images}

画像を埋め込む際は {\tt ![]()} を使います

\begin{itembox}[c]{画像埋め込み例}
\begin{verbatim}
![](/path/to/image.eps)
:caption <caption> :scale <scale> :label <label>
\end{verbatim}
\end{itembox}

\subsection{Horizontal Rules}

ハイフンかアスタリスクを3つ以上並べると水平線が出力されます

\begin{center}
\rule{3in}{0.4pt}
\end{center}

\begin{center}
\rule{3in}{0.4pt}
\end{center}

\end{document}