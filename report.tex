\documentclass[a4j]{jarticle}
\usepackage{amsmath,amssymb} % 数式
\usepackage{fancybox,ascmac} % 丸枠
\usepackage[dvipdfmx]{graphicx} % 図
% プログラムリストで使用
\usepackage{ascmac}
\usepackage{here}
\usepackage{txfonts}
\usepackage{listings, jlisting} % プログラムリスト
\renewcommand{\lstlistingname}{リスト}
\lstset{
  language=c,
  basicstyle=\ttfamily\normalsize, % コードのフォントと文字サイズ
  commentstyle=\textit, % コメント部分のフォント
  classoffset=1,
  keywordstyle=\bfseries,
  frame=tRBl,
  framesep=5pt,
  showstringspaces=false,
  numbers=left,
  stepnumber=1,
  numberstyle=\footnotesize,
  tabsize=3 % インデントの深さ(スペースの数)
}

\begin{document}
\section{はじめに}

これはMarkdownファイルをtexファイルに変換するためのプログラムです

\section{見出し1}

\#\#見出し2
\#\#\#見出し3

\subsection{変換できること}

\begin{itemize}
\item 箇条書き
\item ソースコード(枠で囲むなど)
\item 表
\item 数式
\item 画像の埋め込み
\end{itemize}

\subsection{使用例}

箇条書きには -,+,* が使えます。
リストは 1. のように数字とコロンと1つ以上の空白から始めます。

\begin{itemize}
\item item1
\item item2
\item item3


\begin{itemize}
\item nest1
\item nest2
\item nest3
\end{itemize}
\item item4
\item item5
\end{itemize}

\begin{enumerate}
\item item1
\item item2
\item item3
\item item10
\item item11
\end{enumerate}

ソースコードの出力方法

\begin{itemize}
\item ソースコードの前後に1つ以上の空行を置く
\item 4つ以上のインデントまたは1つ以上のタブを置く
\item {\tt :caption} でタイトルを付ける
\item {\tt :label} でラベルを付ける
\item {\tt :listing} で行番号と改ページを行う枠に変更する
\end{itemize}

\begin{itembox}[c]{ソースコード1}\begin{verbatim}
p "hello world"
\end{verbatim}\end{itembox}

リスト\ref{sample1}に繰り返し処理の例を示します

\begin{lstlisting}[caption=繰り返しの例 ,label=sample1]
(1..10).each do |i|
	p i
end
\end{lstlisting}

\begin{table}[h]
\centering
\caption{表の説明 }
\label{table:1}
\begin{tabular}{|l|r|c|}
\hline
Left align & Right align & Center align\\
\hline
This & This & This\\
column & column & column\\
will & will & will\\
be & be & be\\
left & right & center\\
aligned & aligned & aligned\\
\hline
\end{tabular}
\end{table}

数式は\$\$で囲みます

\begin{eqnarray*}
\frac{\pi}{2}
= \left( \int_{0}^{\infty} \frac{\sin x}{\sqrt{x}} dx \right)^2 
= \sum_{k=0}^{\infty} \frac{(2k)!}{2^{2k}(k!)^2} \frac{1}{2k+1} 
= \prod_{k=1}^{\infty} \frac{4k^2}{4k^2 - 1}
\end{eqnarray*}

画像を埋め込む際は {\tt ![]()} を使います


\if 0 :caption 画像の説明 :scale 0.6 :label fig:sample1 \fi

ハイフンかアスタリスクを3つ以上並べると水平線が出力されます

\begin{center}
\rule{3in}{0.4pt}
\end{center}

\begin{center}
\rule{3in}{0.4pt}
\end{center}

\end{document}