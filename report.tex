\documentclass[a4j, titlepage]{jarticle}
\usepackage[utf8]{inputenc}
\usepackage{amsmath,amssymb} % 数式
\usepackage{fancybox,ascmac} % 丸枠
\usepackage[dvipdfmx]{graphicx} % 図
\usepackage{verbatim} % ソースコードの埋め込み
% プログラムリストで使用
\usepackage{ascmac}
\usepackage{here}
\usepackage{txfonts}
\usepackage{listings, jlisting} % プログラムリスト
\renewcommand{\lstlistingname}{リスト}
\lstset{
  language=c,
  basicstyle=\ttfamily\small, % コードのフォントと文字サイズ
  commentstyle=\textit, % コメント部分のフォント
  classoffset=1,
  keywordstyle=\bfseries,
  frame=tRBl,
  framesep=5pt,
  showstringspaces=false,
  numbers=left,
  stepnumber=1,
  numberstyle=\footnotesize,
  tabsize=3 % インデントの深さ(スペースの数)
}
\title{ \Huge Markdown -\textgreater{} TeX [ -\textgreater{} PDF ] \\{\LARGE How to write extended markdown and convert to TeX} }
\author{ \Large @TeX2e }
\date{ \Large 2015/4/1 }

\begin{document}
\maketitle
\thispagestyle{empty}
\newpage
\setcounter{page}{1}

% 実装予定
% :preamble do - end で、preambleに追加する項目を設定する
% 
% :preamble do
% 	`\def\lstlistingname{List}`
% 	`\def\tablename{Table}`
% :end

\section{Overview}

これはMarkdownファイルをTeXファイルに変換するためのrubyスクリプトです。

\section{Requirements}

You need to install gem ``kramdown''

\begin{screen}
\begin{verbatim}
$ gem install kramdown
\end{verbatim}
\end{screen}

Or add this line to your Gemfile:

\begin{screen}
\begin{verbatim}
gem 'kramdown'
\end{verbatim}
\end{screen}

then enter {\tt bundle} command to install ``kramdown''

\section{Usage}

enter the following commands:

\begin{screen}
\begin{verbatim}
$ ruby <this_script> [-optsions] <md_file>
\end{verbatim}
\end{screen}

\begin{description}
\item[{\tt \textless{}this\_script\textgreater{}}]\mbox{}\\ mdをtexに変換するスクリプトのファイル名



\item[{\tt \textless{}md\_file\textgreater{}}]\mbox{}\\ 変換元となるmarkdownファイル



\item[{\tt -options}]\mbox{}\\ optionには次のものがあります


\begin{description}
\item[{\tt -p}]\mbox{}\\ texに変換した後、pdfに変換するオプション



\item[{\tt --pdf}]\mbox{}\\ 同上
\end{description}
\end{description}

\section{Structural Elements}

\begin{itemize}
\item Headers
\item Lists
\item Code Blocks
\item Tables
\item Math Blocks
\item Images
\end{itemize}

\section{Markdown Syntax}

\subsection{Headers}

見出しは {\tt \#} を使って表します。

\begin{itembox}[c]{headers example}
\begin{verbatim}
# First level header

## Second level header

### Third level header
\end{verbatim}
\end{itembox}

\subsection{(Un)Ordered Lists}

箇条書きには {\tt -,+,*} が使えます。
リストは {\tt 1.} のように数字とコロンと1つ以上の空白をリストの先頭に付けます。

\begin{itembox}[c]{list example}
\begin{verbatim}
- item1
- item2
- item3

1. item1
2. item2
3. item3
\end{verbatim}
\end{itembox}

Output:

\begin{itemize}
\item item1
\item item2


\begin{itemize}
\item nest1
\item nest2


\begin{itemize}
\item deep nest1
\item deep nest2
\end{itemize}
\end{itemize}
\item item3
\item item4
\end{itemize}

\begin{center}
\rule{3in}{0.4pt}
\end{center}

\begin{enumerate}
\item item1
\item item2


\begin{enumerate}
\item nest1
\item nest2


\begin{enumerate}
\item deep nest1
\item deep nest2
\end{enumerate}
\end{enumerate}
\item item3
\item item4
\end{enumerate}

\subsection{Definition Lists}

定義の次の行に {\tt :} があれば、定義とその説明を書くことができます。

\begin{itembox}[c]{definition list example}
\begin{verbatim}
def1
: description

def2
: description
\end{verbatim}
\end{itembox}

Output:

\begin{description}
\item[Laziness]\mbox{}\\ The quality that makes you go to great effort to reduce overall energy expenditure. It makes you write labor-saving programs that other people will find useful, and document what you wrote so you don't have to answer so many questions about it. Hence, the first great virtue of a programmer.



\item[Impatience]\mbox{}\\ The anger you feel when the computer is being lazy. This makes you write programs that don't just react to your needs, but actually anticipate them. Or at least that pretend to. Hence, the second great virtue of a programmer.



\item[Hubris]\mbox{}\\ Excessive pride, the sort of thing Zeus zaps you for. Also the quality that makes you write (and maintain) programs that other people won't want to say bad things about. Hence, the third great virtue of a programmer.
\end{description}

\subsection{Code Blocks}

ソースコードを出力する方法

\begin{itemize}
\item ソースコードの前後に1つ以上の空行を置く
\item 4つ以上のインデントまたは1つ以上のタブを置く
\item {\tt :caption} でタイトルを付ける
\item {\tt :label} でラベルを付ける
\item {\tt :listing} で行番号と改ページを行う枠に変更する
\end{itemize}

ソースコードは、丸枠で囲むか、行番号付きの枠で囲むかの2通りの選択肢があります。

コードを説明なしの丸枠で囲む場合は、前後に何も書きません。

\begin{itembox}[c]{without caption}
\begin{verbatim}
 
    printf("hello, world");
 
\end{verbatim}
\end{itembox}

Output:

\begin{screen}
\begin{verbatim}
printf("hello, world");
\end{verbatim}
\end{screen}

\begin{center}
\rule{3in}{0.4pt}
\end{center}

コードを丸枠で囲む場合は、コードから2行上に {\tt :caption} から始まる行を書きます。

\begin{itembox}[c]{丸枠の例}
\begin{verbatim}
:caption <caption>

    printf("hello, world");
 
\end{verbatim}
\end{itembox}

Output:

\begin{itembox}[c]{hello, world}
\begin{verbatim}
printf("hello, world");
\end{verbatim}
\end{itembox}

\begin{center}
\rule{3in}{0.4pt}
\end{center}

丸枠に外部ファイルのコードを埋め込む場合は、{\tt :caption} の次の行に、{\tt [embed](\textless{}path\textgreater{})} を書きます。

\begin{itembox}[c]{example to embed code}
\begin{verbatim}
:caption <caption>
    [embed](/path/to/source.c)
 
\end{verbatim}
\end{itembox}

Output:

\begin{itembox}[c]{embed test}
{\small
\verbatiminput{./sample.c}
}
\end{itembox}

\begin{center}
\rule{3in}{0.4pt}
\end{center}

行番号付きの枠(リスト)を使う方法は2通りあります。

コードをリストにする場合は、コードから3行上に {\tt :caption} と {\tt :label}を書き、2行上に {\tt :listing} を書きます。

\begin{itembox}[c]{行番号付きの枠}
\begin{verbatim}
:caption <caption> :label <label>
:listing

    (1..10).each do |i|
        p i
    end
 
\end{verbatim}
\end{itembox}

Output:
 
\begin{lstlisting}[caption=iterate ,label=list:1]
(1..10).each do |i|
	p i
end
\end{lstlisting}

\begin{center}
\rule{3in}{0.4pt}
\end{center}

リストに外部ファイルのコードを埋め込む場合は、{\tt :listing} の次の行に、{\tt [embed](\textless{}path\textgreater{})} を書きます。

\begin{itembox}[c]{example to embed code in listing}
\begin{verbatim}
:caption <caption> :label <label>
:listing
    [embed](/path/to/source.c)
 
\end{verbatim}
\end{itembox}

Output:

\lstinputlisting[caption=embed in listing ,label=list:2]
{./sample.c}

\begin{center}
\rule{3in}{0.4pt}
\end{center}

リストの場合、{\tt :}{\tt ref\{\textless{}label\textgreater{}\}} で参照を行うことができます。

List \ref{list:2} shows \ldots{}

\subsection{Tables}

To display table, we use pipe {\tt \textbar{}} and minus {\tt -}

\begin{itembox}[c]{table example}
\begin{verbatim}
:caption <caption> :label <label>

 colum1     | colum2      | colum3
:-----------|------------:|:------------:
 This       | This        | This         
 column     | column      | column       
 will       | will        | will         
 be         | be          | be           
 left       | right       | center       
 aligned    | aligned     | aligned   
\end{verbatim}
\end{itembox}

\begin{table}[h]
\centering
\caption{table sample }
\label{table:1}
\begin{tabular}{|l|r|c|}
\hline
Left align & Right align & Center align\\
\hline
This & This & This\\
column & column & column\\
will & will & will\\
be & be & be\\
left & right & center\\
aligned & aligned & aligned\\
\hline
\end{tabular}
\end{table}

表の場合、{\tt :}{\tt ref\{\textless{}label\textgreater{}\}} で参照を行うことができます。

Table \ref{table:1} shows \ldots{}

\subsection{Math Blocks}

数式は\$\$で囲みます

\begin{itembox}[c]{equation example}
\begin{verbatim}
$$ inline math block $$

$$
multiline math block
$$
\end{verbatim}
\end{itembox}

Output:

解の公式は $x = \frac{-b\pm\sqrt{b^2-4ac}}{2a}$ で表せます。
式 $\sum_{n = 1}^{\infty} \frac{1}{n}$ の収束値を求めます。

\begin{displaymath}
\frac{\pi}{2}
= \left( \int_{0}^{\infty} \frac{\sin x}{\sqrt{x}} dx \right)^2 
= \sum_{k=0}^{\infty} \frac{(2k)!}{2^{2k}(k!)^2} \frac{1}{2k+1} 
= \prod_{k=1}^{\infty} \frac{4k^2}{4k^2 - 1}
\end{displaymath}

\subsection{Images}

画像を埋め込む際は {\tt ![]()} を使います

\begin{itembox}[c]{example of displaying an image}
\begin{verbatim}
![](/path/to/image.eps)
:caption <caption> :scale <scale> :label <label>
\end{verbatim}
\end{itembox}

\subsection{Horizontal Rules}

ハイフンかアスタリスクを4つ以上並べると水平線が出力されます

\begin{itembox}[c]{horizontal rules example}
\begin{verbatim}
----

****
\end{verbatim}
\end{itembox}

Output:

\begin{center}
\rule{3in}{0.4pt}
\end{center}

\begin{center}
\rule{3in}{0.4pt}
\end{center}

\subsection{Blockquotes}

\begin{itembox}[c]{blockquotes example}
\begin{verbatim}
> This is a blockquote
> on multiple line
> but it looks one line
\end{verbatim}
\end{itembox}

Output:

This is para text.

\begin{quote}
This is a blockquote
on multiple line.
but it looks like one line
\end{quote}

List work in blockquotes

\begin{quotation}
This is a blockquote

\begin{itemize}
\item list work
\item item1
\item item2
\end{itemize}
\end{quotation}

\subsection{Footnotes}

This is some text.\footnote{This is \emph{italic} footnote.}. Other text.\footnote{You can use blockquotes.

\begin{quote}
Blockquotes can be in a footnote.
\end{quote}}.

\subsection{Comments}

% This text is completely ignored by kramdown - a comment in the text.

\subsection{Alias}


(hoge fuga piyo)

\end{document}